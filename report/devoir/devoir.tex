
\documentclass[letterpaper]{article}
\usepackage[margin=1in]{geometry}

\usepackage[utf8]{inputenc}
\usepackage{lmodern}
\usepackage[T1]{fontenc}
\usepackage[francais]{babel}

\title{Plan de travail \\ Teach and repeat}
\author{David Landry, Alexandre Gariépy}

\begin{document}
\maketitle

Pour les besoins du cours on a cru bon de faire une remise partielle dans des délais
raisonnables. Cependant l'algorithme est loin d'être terminé: voici trois façons dont on pourrait
l'améliorer encore plus.

\section{Gestion des rotations}

\section{Réoptimisation du graphe}

Le sous-ensemble de points de référence qu'on crée est considéré comme optimal, mais en fait on
pourrait faire encore mieux. Pendant le repeat, on utilise un point de référence pour se localiser
jusqu'à ce que le prochain point de référence devienne plus proche que l'ancien. Ensuite on fait la
transition. Cependant, lors de l'optimisation on montre que les points de référence peuvent nous
localiser comme il faut jusqu'au prochain point de référence. C'est comme si pendant le repeat on
changeait de point de référence seulement quand on a atteint le prochain point, au lieu de le faire
quand il devient plus proche.

En d'autres mots, les cartes créées sont deux fois plus robustes que demandé. On peut se servir de
cela pour optimiser encore plus les cartes créées.

\section{Expérimentaions avec le husky}

On devrait essayer une des cartes optimisées dans le cadre d'un teach and repeat réel, pour montrer
que les cartes nous permettent réellement de nous localiser sans problème. Exemples de choses que
l'expérience pourrait montrer:

\begin{itemize}
\item Que les cartes optimisées sont effectivement suffisantes pour se localiser.
\item Que les cartes nous donnent une tolérance à l'erreur qui est à peu près équivalente à celle
  qu'on avait demandé lors de l'optimisation de la carte.
\item Plus l'ellipse de convergence demandée est petite, moins le repeat est robuste.
\end{itemize}

Pour pouvoir faire cette expérience, il faudra apporter quelques modifications à l'implémentation
actuelle du teach and repeat.

\begin{itemize}
\item Les cartes dans le teach and repeat sont légèrement différentes des cartes manipulées pendant
  l'optimisation. Il faudra uniformiser le tout dans le code.
\end{itemize}



\end{document}
